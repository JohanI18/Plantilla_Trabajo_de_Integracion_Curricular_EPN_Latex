%%%%%%%%%%%%%%%%%%%%%%%%%%%%%%%%%%%%%%%%%%%%%%%%%%%%%%%%%%%%%%%%%%%%%%%%%%%%%%%%
%   Sección 2
%
%%%%%%%%%%%%%%%%%%%%%%%%%%%%%%%%%%%%%%%%%%%%%%%%%%%%%%%%%%%%%%%%%%%%%%%%%%%%%%%%
\chapter{\texorpdfstring{METODOLOGÍA}{Sección 2: Metodología}}
\label{chap:metodologia} % Added a label for potential cross-referencing

% Introducción a la Metodología

Describir el diseño o el planteamiento que ha sido utilizado para el desarrollo del componente, el cual depende del método seleccionado (hipotético-deductivo, inductivo, entre otros). Se sugiere incluir, los que correspondan:
    • Enfoque (cualitativo, cuantitativo o mixto).
    • Tipo de trabajo: exploratorio, descriptivo, explicativo, experimental, estudio de casos, entre otros.
    • Técnica de recolección de información (entrevistas, cuestionarios, análisis documental, entre otras).
    • Técnica de análisis de la información.
    • Uso de herramientas de IA generativa.
Este capítulo debe incluir toda la información necesaria para que un interesado pueda replicar el componente sin dificultades. Se debe mencionar explícitamente cuáles actividades se realizaron para cumplir con los objetivos planteados. 

Sugerencias: Esta sección debería abarcar un 50% de la extensión del documento. 

%%%%%%%%%%%%%%%%%%%%%%%%%%%%%%%%%%%%%%%%%%%%%%%%%%%%%%%%%%%%%%%%%%%%%%%%%%%%%%%%
%   SECCIÓN 2.1
%%%%%%%%%%%%%%%%%%%%%%%%%%%%%%%%%%%%%%%%%%%%%%%%%%%%%%%%%%%%%%%%%%%%%%%%%%%%%%%%
