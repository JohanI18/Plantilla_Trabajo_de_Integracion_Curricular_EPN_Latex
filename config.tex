%%%%%%%%%%%%%%%%%%%%%%%%%%%%%%%%%%%%%%%%%%%%%%%%%%%%%%%%%%%%%%%%%%%%%%%%%%%%%%%%
%   CONFIGURACIONES GENERALES
%%%%%%%%%%%%%%%%%%%%%%%%%%%%%%%%%%%%%%%%%%%%%%%%%%%%%%%%%%%%%%%%%%%%%%%%%%%%%%%%

%%%%%%%%%%%%%%%%%%%%%%%%%%%%%%%%%%%%%%%%%%%%%%%%%%%%%%%%%%%%%%%%%%%%%%%%%%%%%%%%
%   Formato de Hoja e Idioma
%%%%%%%%%%%%%%%%%%%%%%%%%%%%%%%%%%%%%%%%%%%%%%%%%%%%%%%%%%%%%%%%%%%%%%%%%%%%%%%%
\listfiles
% Letra a 11pt, Papel INEN A4  SE REALIZO CAMBIO
\documentclass[11pt,a4paper]{report}
% Documento escrito en Español
\usepackage[activeacute,spanish]{babel}
% Caracteres en Español
\usepackage[utf8]{inputenc}
% Otros caracteres Español
\usepackage[T1]{fontenc}
% Márgenes de página
\usepackage[
    a4paper,
    left=3cm,
    top=3cm,
    right=2.5cm,
    bottom=2.5cm,
    footskip=1.5cm]{geometry} 
% Tiempo (útil para ver fecha y hora de compilación)
\usepackage{datetime}
% Formato de Fecha de Caratula *personalizado
\newdateformat{epnDate}{\monthname[\THEMONTH] \THEYEAR} 
% Punto decimal - Ejemplo: 2.54
\spanishdecimal{.}


%%%%%%%%%%%%%%%%%%%%%%%%%%%%%%%%%%%%%%%%%%%%%%%%%%%%%%%%%%%%%%%%%%%%%%%%%%%%%%%%
%  Estilo bibliografía
%%%%%%%%%%%%%%%%%%%%%%%%%%%%%%%%%%%%%%%%%%%%%%%%%%%%%%%%%%%%%%%%%%%%%%%%%%%%%%%%
\usepackage[style=ieee]{biblatex}
\addbibresource{bibliografia.bib}

%%%%%%%%%%%%%%%%%%%%%%%%%%%%%%%%%%%%%%%%%%%%%%%%%%%%%%%%%%%%%%%%%%%%%%%%%%%%%%%%
%  Quotes
%%%%%%%%%%%%%%%%%%%%%%%%%%%%%%%%%%%%%%%%%%%%%%%%%%%%%%%%%%%%%%%%%%%%%%%%%%%%%%%%
\usepackage{csquotes}
\usepackage{url}
\urlstyle{same}
%%%%%%%%%%%%%%%%%%%%%%%%%%%%%%%%%%%%%%%%%%%%%%%%%%%%%%%%%%%%%%%%%%%%%%%%%%%%%%%%
%   Letra e Interlineado
%%%%%%%%%%%%%%%%%%%%%%%%%%%%%%%%%%%%%%%%%%%%%%%%%%%%%%%%%%%%%%%%%%%%%%%%%%%%%%%%
% Paquete de fuentes (Arial, Times, Courier)
\usepackage{helvet}
% Arial para el cuerpo de texto
\renewcommand{\familydefault}{\sfdefault}
% Interlineado
\linespread{1.5}
% Fuentes 24pt *personalizada
%   Ejemplo de uso: {\sizeveinticuatro Palabras con tamaño veinticuatro} 
\newcommand{\sizeveinticuatro}{\fontsize{24pt}{20pt}\selectfont}
% Fuentes 16pt *personalizada
%   Ejemplo de uso: {\sizedieciseis Palabras con tamaño dieciseis}
\newcommand{\sizedieciseis}{\fontsize{16pt}{20pt}\selectfont} 
% Fuentes 14pt *personalizada
%   Ejemplo de uso: {\sizecatorce Palabras con tamaño catorce}
\newcommand{\sizecatorce}{\fontsize{14}{20pt}\selectfont}
% Fuentes 12pt *personalizada
%   Ejemplo de uso: {\sizedoce Palabras con tamaño doce}
\newcommand{\sizedoce}{\fontsize{12}{20pt}\selectfont}
% Sin indentado (primera línea francesa) para párrafos
\setlength\parindent{0pt}


%%%%%%%%%%%%%%%%%%%%%%%%%%%%%%%%%%%%%%%%%%%%%%%%%%%%%%%%%%%%%%%%%%%%%%%%%%%%%%%%
%   Cabecera y Pie de Página
%%%%%%%%%%%%%%%%%%%%%%%%%%%%%%%%%%%%%%%%%%%%%%%%%%%%%%%%%%%%%%%%%%%%%%%%%%%%%%%%
% Paquete de manipulación de Cabeceras y Pies de Página
\usepackage{fancyhdr}
% Estilo plano
\pagestyle{fancyplain}
% Control de Cabecera y Pie de Página 
\fancyhf{}
% Colocación de números de página en parte inferior centrado
\fancyfoot[C]{\thepage}
% Sin línea de Cabecera 
\renewcommand{\headrulewidth}{0pt}
% Corrige la posición del número en la Cabecera 
\geometry{headheight=15pt}
% Activar Interlineado para pie de página 
\usepackage{setspace}
% Mismo alto de notas al pie de página desde el final de la hoja 
\flushbottom
% Interlineado entre notas al pie de página 
\setlength{\footnotesep}{0.4cm}
% Espacio entre cuerpo de texto y pie de página 
\setlength{\skip\footins}{1.1cm}
% Utilidades Justificado de párrafos
\usepackage{ragged2e}
% Utilidad para estilo de marcas de pie de página
\usepackage{scrextend}
% Estilo de numero de footcite en texto
\deffootnotemark{\textsuperscript{[\thefootnotemark]}}
% Estilo de número de footcite al pie de la página
\deffootnote[3em]{3em}{0em}{
    \parbox[b][\height][r]{2.3em}
    {\footnotesize\textsuperscript{[\thefootnotemark]}}
    \enskip
}



%%%%%%%%%%%%%%%%%%%%%%%%%%%%%%%%%%%%%%%%%%%%%%%%%%%%%%%%%%%%%%%%%%%%%%%%%%%%%%%%
%   Formato de Títulos
%%%%%%%%%%%%%%%%%%%%%%%%%%%%%%%%%%%%%%%%%%%%%%%%%%%%%%%%%%%%%%%%%%%%%%%%%%%%%%%%
% Paquete de manipulación de Títulos
\usepackage{titlesec}
%   Capítulos
\titleformat{\chapter}[hang] 
{\bfseries\sizedieciseis}
{\MakeUppercase{}\ \thechapter}
{5.0mm}{\sizedieciseis\MakeUppercase}
%   Subcapítulos 1
\titleformat{\section}[hang]  
{\bfseries\sizecatorce}
{\thesection }
{5.0mm}{\sizecatorce\MakeUppercase}
%   Subcapítulos 2
\titleformat{\subsection}[hang]  
%   SE HIZO CAMBIO TAMAÑO LETRA y se quito uppercase
{\bfseries\sizecatorce}
{\thesubsection }
{5.0mm}{\sizecatorce}
%   Subcapítulos 3
\titleformat{\subsubsection}[hang]  
%   SE HIZO CAMBIO TAMAÑO LETRA
{\bfseries\sizedoce}
{\thesubsubsection }
{5.0mm}{\sizedoce}
%   Subcapítulos 4
\titleformat{\paragraph}[hang]
{\em\sizedoce}
{\theparagraph} 
{5.0mm}{\sizedoce}
%   Otros Títulos
\newcommand{\titulos}{\sf\bf\sizecatorce\centerline}
\newcommand{\titulosizq}{\sf\bf\sizecatorce}


%%%%%%%%%%%%%%%%%%%%%%%%%%%%%%%%%%%%%%%%%%%%%%%%%%%%%%%%%%%%%%%%%%%%%%%%%%%%%%%%
%   Profundidad de la Numeración
%%%%%%%%%%%%%%%%%%%%%%%%%%%%%%%%%%%%%%%%%%%%%%%%%%%%%%%%%%%%%%%%%%%%%%%%%%%%%%%%
 % Desde Capítulos hasta Subcapítulos 4
\setcounter{secnumdepth}{3}
 % Profundidad de la Tabla de Contenido
\setcounter{tocdepth}{2}




%%%%%%%%%%%%%%%%%%%%%%%%%%%%%%%%%%%%%%%%%%%%%%%%%%%%%%%%%%%%%%%%%%%%%%%%%%%%%%%%
%   ¡¡¡ IMPORTANTE !!!
%   Hiperenlaces e Información para el PDF generado
%
%   Cambiar la información de pdfauthor, pdftitle, pdfsubject y pdfcreator
%   de acuerdo al proyecto de titulación
%%%%%%%%%%%%%%%%%%%%%%%%%%%%%%%%%%%%%%%%%%%%%%%%%%%%%%%%%%%%%%%%%%%%%%%%%%%%%%%%
\usepackage{float} % Posición flotante (ej. imágenes) 
\usepackage[breaklinks,hypertexnames=false]{hyperref} 
\hypersetup{ %hiperenlaces
    pdfauthor={Nombre},
    pdftitle={Título},
    pdfsubject={Descripción},
    pdfkeywords={Keyword1, keyword2},
    colorlinks, 
    citecolor=black, 
    filecolor=black, 
    linkcolor=black, 
    urlcolor=black 
}
% Los hiperenlaces apuntan a las figuras, no a los nombres de las figuras 
\usepackage[all]{hypcap}


%%%%%%%%%%%%%%%%%%%%%%%%%%%%%%%%%%%%%%%%%%%%%%%%%%%%%%%%%%%%%%%%%%%%%%%%%%%%%%%%
%   Notación URL comentado
%%%%%%%%%%%%%%%%%%%%%%%%%%%%%%%%%%%%%%%%%%%%%%%%%%%%%%%%%%%%%%%%%%%%%%%%%%%%%%%%
%\usepackage{url}
%\makeatletter
%\g@addto@macro{\UrlBreaks}{
%\UrlOrds\do\-\do\_\do\/\do\:\do\2\do\3\do\4\do\5\do\6\do\7\do\8\do\9\do\0\do\&}
%\makeatother


%%%%%%%%%%%%%%%%%%%%%%%%%%%%%%%%%%%%%%%%%%%%%%%%%%%%%%%%%%%%%%%%%%%%%%%%%%%%%%%%
%   Formato de la Tabla de Contenido y Nombres en Español
%%%%%%%%%%%%%%%%%%%%%%%%%%%%%%%%%%%%%%%%%%%%%%%%%%%%%%%%%%%%%%%%%%%%%%%%%%%%%%%%
\usepackage[titles]{tocloft}
% Separación antes del titulo
\setlength\cftbeforetoctitleskip{0pt}
% Separación después del titulo
\setlength\cftaftertoctitleskip{1cm}
\renewcommand\cftchappresnum{\chaptername\space}
% Formato: Capítulo # - Título del Capítulo 
\renewcommand\cftchappresnum{ }
% Separación del Número de Capítulo
\setlength{\cftchapnumwidth}{2em}
\newcommand\centrarcelda[1]{\let\temp=\\%
  #1%
    \let\\=\temp
}


%%%%%%%%%%%%%%%%%%%%%%%%%%%%%%%%%%%%%%%%%%%%%%%%%%%%%%%%%%%%%%%%%%%%%%%%%%%%%%%%
%   Formato de Tablas 
%%%%%%%%%%%%%%%%%%%%%%%%%%%%%%%%%%%%%%%%%%%%%%%%%%%%%%%%%%%%%%%%%%%%%%%%%%%%%%%%
\usepackage{array}
\usepackage{calc}
\usepackage[table]{xcolor}
\usepackage{booktabs}
\usepackage{tabulary}
% Tablas que ocupan más de una página
\usepackage{longtable}
% Para cambiar interlineado del texto de la tabla
\usepackage{setspace}
\setlength\tymin{5cm}
 % Varias filas
\usepackage{multirow}
\definecolor{bluetable}{RGB}{175,198,233}
 % Color de líneas de Tablas
\arrayrulecolor{bluetable}
 % Grosor de líneas de Tablas
\setlength{\arrayrulewidth}{.9pt}
\usepackage{hhline}
% Márgenes personalizados en landscape
\makeatletter
\def\fudge#1#2{%
\addtolength\textheight{#1}%
\@colroom\textheight
\vsize\textheight
\@colht\textheight
\def\LS@rot{%
  \setbox\@outputbox\vbox{\hbox{\kern-#2\rotatebox{90}{\box\@outputbox}}}}%
\clearpage}
\makeatother
% Tamaño de letra small para tablas
\usepackage{etoolbox}
\AtBeginEnvironment{longtable}{\small}
\AtBeginEnvironment{tabular}{\small}


%%%%%%%%%%%%%%%%%%%%%%%%%%%%%%%%%%%%%%%%%%%%%%%%%%%%%%%%%%%%%%%%%%%%%%%%%%%%%%%%
%   Hojas Horizontales
%%%%%%%%%%%%%%%%%%%%%%%%%%%%%%%%%%%%%%%%%%%%%%%%%%%%%%%%%%%%%%%%%%%%%%%%%%%%%%%%
% Posición Horizontal para ciertas páginas * útil para tablas largas
\usepackage{pdflscape}


%%%%%%%%%%%%%%%%%%%%%%%%%%%%%%%%%%%%%%%%%%%%%%%%%%%%%%%%%%%%%%%%%%%%%%%%%%%%%%%%
%   Representar Código Fuente
%%%%%%%%%%%%%%%%%%%%%%%%%%%%%%%%%%%%%%%%%%%%%%%%%%%%%%%%%%%%%%%%%%%%%%%%%%%%%%%%
% Necesario para la inclusión de código fuente 
\usepackage{listings}
% Necesario para resaltado de la sintaxis
\usepackage{color}
% Colores de resaltado 
\definecolor{letraAzul}{cmyk}{1,0.5,0,0.5}
\definecolor{lstrule}{RGB}{158,180,204}
\definecolor{fondo}{RGB}{245,245,250}
\definecolor{gray}{rgb}{0.5,0.5,0.5}
\definecolor{darkviolet}{rgb}{0.5,0,0.4}
\definecolor{darkpink}{rgb}{0.8,0.3,0.5}
\lstset{ %
  language=Java,
  basicstyle=\footnotesize\color{letraAzul},
  numbers=left,
  numberstyle=\scriptsize\color{gray},
  numberfirstline=true,
  firstnumber=1,
  stepnumber=5,
  numbersep=8pt,
  backgroundcolor=\color{fondo},
  showspaces=false,
  showstringspaces=false,
  showtabs=false,
  frame=single,
  rulecolor=\color{lstrule},
  tabsize=4,
  captionpos=t,
  breaklines=true,
  breakatwhitespace=false,
  title=\lstname,
  keywordstyle=\bfseries\color{darkviolet},
  commentstyle=\color{gray},
  stringstyle=\color{darkpink},
  escapeinside={\%*}{*)},
  morekeywords={*,...}
  inputencoding=utf8,
  emphstyle=\color{red},
  extendedchars=true,
  literate={á}{{\'a}}1
         {é}{{\'e}}1
         {í}{{\'i}}1
         {ó}{{\'o}}1
         {ú}{{\'u}}1
         {ñ}{{\~n}}1
}
% !Hack para estilo
\makeatletter
\gdef\lst@SkipOrPrintLabel{%
    \ifnum\lst@skipnumbers=\z@
        \global\advance\lst@skipnumbers-\lst@stepnumber\relax
        \lst@PlaceNumber
        \lst@numberfirstlinefalse
    \else
        \lst@ifnumberfirstline
            {\def\thelstnumber{\@arabic\c@lstnumber}\lst@PlaceNumber}%
            \lst@numberfirstlinefalse
        \else
            {\def\thelstnumber{-}\lst@PlaceNumber}%
        \fi
    \fi
\global\advance\lst@skipnumbers\@ne}%


%%%%%%%%%%%%%%%%%%%%%%%%%%%%%%%%%%%%%%%%%%%%%%%%%%%%%%%%%%%%%%%%%%%%%%%%%%%%%%%%
%   Imágenes
%%%%%%%%%%%%%%%%%%%%%%%%%%%%%%%%%%%%%%%%%%%%%%%%%%%%%%%%%%%%%%%%%%%%%%%%%%%%%%%%
\usepackage{fancybox}
% Paquete de manipulación de Imágenes
\usepackage{graphicx,type1cm,eso-pic}
\usepackage[
    font=small,
    format=plain,
    labelfont=bf,up,
    justification=default,
    compatibility=false]{caption}
% Letras distintas para los captions de las imágenes
% Hyphenation con texttt
\newcommand*\justifyCaption{%
  \fontdimen2\font=0.4em% spacio entre líneas
  \fontdimen3\font=0.2em% ajuste entre líneas
  \fontdimen4\font=0.1em% ajuste entre líneas
  \fontdimen7\font=0.1em% espacio extra
  \hyphenchar\font=`\-% permitir hyphenation
}
% Fin Hyphenation con texttt

\definecolor{ShadowColor}{RGB}{128,128,128}

 % ShadowBox para páginas falsas
\makeatletter
\newcommand\Cshadowbox{\VerbBox\@Cshadowbox}
\def\@Cshadowbox#1{%
  \setbox\@fancybox\hbox{\fbox{#1}}%
  \leavevmode\vbox{%
    \offinterlineskip
    \dimen@=\shadowsize
    \advance\dimen@ .5\fboxrule
    \hbox{\copy\@fancybox\kern.5\fboxrule\lower\shadowsize\hbox{%
      \color{ShadowColor}\vrule \@height\ht\@fancybox \@depth\dp\@fancybox 
\@width\dimen@}}%
    \vskip\dimexpr-\dimen@+0.5\fboxrule\relax
    \moveright\shadowsize\vbox{%
      \color{ShadowColor}\hrule \@width\wd\@fancybox \@height\dimen@}}}
\makeatother

 % subfloats
\usepackage[list=true]{subcaption}


%%%%%%%%%%%%%%%%%%%%%%%%%%%%%%%%%%%%%%%%%%%%%%%%%%%%%%%%%%%%%%%%%%%%%%%%%%%%%%%%
%  Marca de Agua para Borrador 
%%%%%%%%%%%%%%%%%%%%%%%%%%%%%%%%%%%%%%%%%%%%%%%%%%%%%%%%%%%%%%%%%%%%%%%%%%%%%%%%
%!!! Quitar cuando sea entregable final
%\makeatletter
%\AddToShipoutPicture{%
%\setlength{\@tempdimb}{.5\paperwidth}%
%\setlength{\@tempdimc}{.5\paperheight}%
%\setlength{\unitlength}{1pt}%
%\put(\strip@pt\@tempdimb,\strip@pt\@tempdimc){%
%\makebox(-50,200){\rotatebox{45}{\textcolor[gray]{0.95}%
% Alterar
%{\fontsize{3cm}{3cm}\selectfont{Borrador}}}}%
%\makebox(50,0){\rotatebox{45}{\textcolor[gray]{0.95}%
%{\fontsize{2cm}{2cm}\selectfont{\today}}}}
%\makebox(-600,-0){\rotatebox{90}{\textcolor[gray]{0.95}%
%{\fontsize{0.7cm}{0.7cm}\selectfont{\textcopyright 
%Copyright 2012 - Paúl Gualotuña - EPN / Facultad de Ingeniería de Sistemas}}}}
%}%
%}
%\makeatother

%%%%%%%%%%%%%%%%%%%%%%%%%%%%%%%%%%%%%%%%%%%%%%%%%%%%%%%%%%%%%%%%%%%%%%%%%%%%%%%%
%   Glosario de términos y acrónimos 
%%%%%%%%%%%%%%%%%%%%%%%%%%%%%%%%%%%%%%%%%%%%%%%%%%%%%%%%%%%%%%%%%%%%%%%%%%%%%%%%
 % Paquete necesario reemplazo de datatool-base.sty
\usepackage{datatool}
 %%%% Los paquetes glossaries y xindy deben estar instalados
\usepackage[toc,acronym,xindy]{glossaries}
% Mantener acrónimos en versalitas (la opción smallcaps fue removida en versiones nuevas)
\renewcommand*{\acronymfont}[1]{\textsc{#1}}
\makeglossaries
 % Convierte la primera letra de las palabras a Mayúscula
\usepackage{mfirstuc}
 % Primera letra de palabras de "glosario" a Mayúsculas
\renewcommand{\glsnamefont}[1]{\makefirstuc{#1}}
 % Glosario en forma de árbol indizado
\usepackage{glossary-super}
 % Generar Glosario


 % Enlace al documento de glosarios * archivo: glosario.tex
%\input{glosario}


%%%%%%%%%%%%%%%%%%%%%%%%%%%%%%%%%%%%%%%%%%%%%%%%%%%%%%%%%%%%%%%%%%%%%%%%%%%%%%%%
%%%%%%%%%%%%%%%%%%%%%%%%%%%%%%%%%%%%%%%%%%%%%%%%%%%%%%%%%%%%%%%%%%%%%%%%%%%%%%%%
%   ¡¡¡ HACKS !!! 
%%%%%%%%%%%%%%%%%%%%%%%%%%%%%%%%%%%%%%%%%%%%%%%%%%%%%%%%%%%%%%%%%%%%%%%%%%%%%%%%
%%%%%%%%%%%%%%%%%%%%%%%%%%%%%%%%%%%%%%%%%%%%%%%%%%%%%%%%%%%%%%%%%%%%%%%%%%%%%%%%

%%%%%%%%%%%%%%%%%%%%%%%%%%%%%%%%%%%%%%%%%%%%%%%%%%%%%%%%%%%%%%%%%%%%%%%%%%%%%%%%
%   Listas Enumeradas
%%%%%%%%%%%%%%%%%%%%%%%%%%%%%%%%%%%%%%%%%%%%%%%%%%%%%%%%%%%%%%%%%%%%%%%%%%%%%%%%
% Modificar las listas enumeradas
\usepackage{enumitem}
% litem permite crear un titulo en negrilla para elementos de 
% una lista - Ejemplo de uso: 
%   \begin{enumerate}
%       \litem{Título de elemento de la lista 1.-}
%       \litem{Título de elemento de la lista 2.-}
%       \litem{Título de elemento de la lista 3.-}
%   \end{enumerate}
\newcommand\litem[1]{\item{\bfseries #1\enspace}}


%%%%%%%%%%%%%%%%%%%%%%%%%%%%%%%%%%%%%%%%%%%%%%%%%%%%%%%%%%%%%%%%%%%%%%%%%%%%%%%%
%   Indentación extra
%%%%%%%%%%%%%%%%%%%%%%%%%%%%%%%%%%%%%%%%%%%%%%%%%%%%%%%%%%%%%%%%%%%%%%%%%%%%%%%%
% myindent permite añadir indentación extra personalizada a 
% cualquier bloque de texto deseado - Ejemplo de uso:
%   \begin{myindent}{1cm}
%       Este es un cuerpo de texto que va a ser indentado 1cm.
%
%       Este es otro cuerpo de texto que sigue indentado 1cm.
%   \end{myindent}
\newenvironment{myindent}[1]
{\begin{list}{}{\setlength{\leftmargin}{#1}}\item[]}{\end{list}}


%%%%%%%%%%%%%%%%%%%%%%%%%%%%%%%%%%%%%%%%%%%%%%%%%%%%%%%%%%%%%%%%%%%%%%%%%%%%%%%%
%   Viñetas personalizadas
%%%%%%%%%%%%%%%%%%%%%%%%%%%%%%%%%%%%%%%%%%%%%%%%%%%%%%%%%%%%%%%%%%%%%%%%%%%%%%%%
 % Símbolos para viñetas
\usepackage{pifont}
\renewcommand{\labelitemi}{\ding{112}}
\renewcommand{\labelitemii}{\ding{71}}


%%%%%%%%%%%%%%%%%%%%%%%%%%%%%%%%%%%%%%%%%%%%%%%%%%%%%%%%%%%%%%%%%%%%%%%%%%%%%%%%
%   Referencias cruzadas
%%%%%%%%%%%%%%%%%%%%%%%%%%%%%%%%%%%%%%%%%%%%%%%%%%%%%%%%%%%%%%%%%%%%%%%%%%%%%%%%
\newcommand{\fullrefuno}[1]{(véase \ref{#1}, pág. \pageref{#1})}
\newcommand{\fullref}[2]{en \ref{#1}, pág. \pageref{#2}}
\newcommand{\refdos}[2]{(véase \ref{#1} y  \ref{#2})}
\newcommand{\fullreffig}[1]{Fig.~\ref{#1} pág. \pageref{#1}} 
\newcommand{\fullreftab}[1]{Tab.~\ref{#1} pág. \pageref{#1}} 
\newcommand{\fullrefcod}[1]{Cód.~\ref{#1} pág. \pageref{#1}} 
\newcommand{\fullrefanx}[1]{(véase el Anexo~\ref{#1})} 


%%%%%%%%%%%%%%%%%%%%%%%%%%%%%%%%%%%%%%%%%%%%%%%%%%%%%%%%%%%%%%%%%%%%%%%%%%%%%%%%
%   Adjuntar archivos en el PDF
%%%%%%%%%%%%%%%%%%%%%%%%%%%%%%%%%%%%%%%%%%%%%%%%%%%%%%%%%%%%%%%%%%%%%%%%%%%%%%%%
%\usepackage{attachfile}
%\usepackage{keyval}
%\usepackage{ifpdf}


%%%%%%%%%%%%%%%%%%%%%%%%%%%%%%%%%%%%%%%%%%%%%%%%%%%%%%%%%%%%%%%%%%%%%%%%%%%%%%%%
%   Separación Especial en Sílabas de palabras
%%%%%%%%%%%%%%%%%%%%%%%%%%%%%%%%%%%%%%%%%%%%%%%%%%%%%%%%%%%%%%%%%%%%%%%%%%%%%%%%
\hyphenation{sa-ffer soft-ware Jesse Nielsen}
%\tolerance=1
%\emergencystretch=\maxdimen
%\hyphenpenalty=10000
%\hbadness=10000


%%%%%%%%%%%%%%%%%%%%%%%%%%%%%%%%%%%%%%%%%%%%%%%%%%%%%%%%%%%%%%%%%%%%%%%%%%%%%%%%
%   Anexos - Appendix 
%%%%%%%%%%%%%%%%%%%%%%%%%%%%%%%%%%%%%%%%%%%%%%%%%%%%%%%%%%%%%%%%%%%%%%%%%%%%%%%%
% Paquete para anexos
\usepackage[toc,title,header]{appendix}
\renewcommand{\appendixname}{Anexo}
\renewcommand{\spanishappendixname}{Anexo}
\renewcommand{\appendixtocname}{Anexos}
\renewcommand{\appendixpagename}{Anexos}

% Hack! para appendix nombre
\makeatletter
\newcommand*\updatechaptername{%
\addtocontents{toc}{\protect\renewcommand*\protect\cftchappresnum{\@chapapp\ }
\setlength{\cftchapnumwidth}{5em} 
}
}
\makeatother

% Hack! Appendix, problemas TOC en PDF
\makeatletter
\appto{\appendices}{\def\Hy@chapapp{Appendix}}
\makeatother

